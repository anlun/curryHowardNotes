\documentclass[sans]{beamer}

\mode<presentation>
{
	% \usetheme{CambridgeUS}
	% \usetheme{Hannover}
	\usetheme{Singapore}
	\usecolortheme{default}
}

\usepackage{cmap}
\usepackage{listings}
\usepackage{lmodern}
\usepackage{color}
\usepackage{minted}
\usepackage{graphicx}
\usepackage{tikz}
\usetikzlibrary{arrows}
\usepackage{wrapfig}

\usepackage[labelformat=empty]{caption}
\usepackage{fontspec}
% \usepackage{polyglossia}
% \setdefaultlanguage{russian}

% \setmainfont[Ligatures=TeX]{DejaVu Serif}
% \setsansfont[Ligatures=TeX]{DejaVu Sans}
% \setmonofont{DejaVu Sans Mono}

\definecolor{myGray}{RGB}{50,50,50}

\begin{document}
\title
[Intuitionistic Logic]
{Intuitionistic Logic}

\author
[JBStuff]{Podkopaev Anton, Daniil Berezun}
\institute{JetBrains Lab}
\date [20-10-14]{20 oct 2014}

\begin{frame}[plain]
	\titlepage
\end{frame}

\begin{frame}{Some history blah-blah}
\end{frame}

\begin{frame}{Examples (1)}
  \emph{
    There is seven $7$'s in a row somewhere in
    the decimal representation
    of the number $\pi$.
  }
\end{frame}

\begin{frame}{Examples (2)}
  \emph{
    There are two irrational numbers $x$ and $y$, such that $x^y$ is rational.
  }
  \vfill
  \begin{block}{Proof:}
    if $\sqrt{2}^{\sqrt{2}}$ is a rational then we can take $x = y = \sqrt{2}$;

    otherwise take $x = \sqrt{2}^{\sqrt{2}}$ and $y = \sqrt{2}$
  \end{block}
\end{frame}

\begin{frame}{Intuitive semantics}
  \emph{BHK-interpretation} (for Brouwler, Heyting, Kolmogorov)

  \begin{block}{Rules:}
    \begin{itemize}
      \item A construction of $\varphi_1 \wedge \varphi_2$ consists of
            a construction of $\varphi_1$ and a construction of $\varphi_2$
      \item A construction of $\varphi_1 \vee \varphi_2$ consists of a
            number $i \in \{1, 2\}$ and a construction of $\varphi_i$
      \item A construction of $\varphi_1 \to \varphi_2$ is a method (function)
            transforming every construction of $\varphi_1$ into a construction
            of $\varphi_2$
      \item There is no possible construction of $\bot$ (where $\bot$ denotes falsity)
    \end{itemize}
  \end{block}
\end{frame}

\begin{frame}{Negation}
  Negation $\lnot \varphi$ is best understood as an abbreviation of an implication $\varphi \to \bot$

  \begin{itemize}
    \item A construction of $\lnot \varphi$ is a method that turns every construction of
      $\varphi$ into a non-existed object
  \end{itemize}
\end{frame}

\begin{frame}{Classic tautologies}
   \begin{enumerate}
     \item $\bot \to p$
     \item $((p \to q) \to p) \to p$
     \item $p \to \lnot \lnot p$
     \item $\lnot \lnot p \to p$
     \item $\lnot \lnot \lnot p \to \lnot p$
     \item $(\lnot q \to \lnot p) \to (p \to q)$
     \item $(p \to q) \to (\lnot q \to \lnot p)$
     \item $\lnot (p \wedge q) \to (\lnot p \vee \lnot q)$
     \item $(\lnot p \vee \lnot q) \to \lnot (p \wedge q)$
     \item $((p \leftrightarrow q) \leftrightarrow r) \leftrightarrow (p \leftrightarrow (q \leftrightarrow r))$
     \item $((p \wedge q) \to r) \leftrightarrow (p \to (q \to r))$
     \item $(p \to q) \leftrightarrow (\lnot p \vee q)$
     \item $\lnot\lnot (p \vee p)$
   \end{enumerate}
\end{frame}

\begin{frame}{Classic tautologies}
   \begin{enumerate}
     \item $\bot \to p$
     \item \textcolor{red}{$((p \to q) \to p) \to p$}
     \item $p \to \lnot \lnot p$
     \item \textcolor{red}{$\lnot \lnot p \to p$}
     \item $\lnot \lnot \lnot p \to \lnot p$
     \item \textcolor{red}{$(\lnot q \to \lnot p) \to (p \to q)$}
     \item $(p \to q) \to (\lnot q \to \lnot p)$
     \item \textcolor{red}{$\lnot (p \wedge q) \to (\lnot p \vee \lnot q)$}
     \item $(\lnot p \vee \lnot q) \to \lnot (p \wedge q)$
     \item \textcolor{red}{$((p \leftrightarrow q) \leftrightarrow r) \leftrightarrow (p \leftrightarrow (q \leftrightarrow r))$}
     \item $((p \wedge q) \to r) \leftrightarrow (p \to (q \to r))$
     \item \textcolor{red}{$(p \to q) \leftrightarrow (\lnot p \vee q)$}
     \item \textcolor{red}{$\lnot\lnot (p \vee p)$}
   \end{enumerate}
\end{frame}

\begin{frame}{Example of failure. $\lnot \lnot p \to p$}
  $\lnot \lnot p \to p = ((p \to \bot) \to \bot) \to p$

  \vspace{1cm}

  You need to construct $p$ from something that takes $p \to \bot$
  and returns $\bot$
\end{frame}

\begin{frame}{The language of logic}
  $\Phi ::= \bot \; | \; PV \; | \; (\Phi \to \Phi) \; |
   \; (\Phi \vee \Phi) \; | \; (\Phi \wedge \Phi) $

  \begin{itemize}
    \item $\lnot \varphi = \varphi \to \bot$
    \item $\varphi \to \psi = (\varphi \to \psi) \wedge (\psi \to \varphi)$
    \item The implication is right associative
  \end{itemize}
\end{frame}

\end{document}
